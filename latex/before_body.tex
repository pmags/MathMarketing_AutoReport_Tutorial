%----------------------------------------------------------------------------------------
%	SETTING UP DOCUMENT
%----------------------------------------------------------------------------------------
\newpage
%~\vfill
%\thispagestyle{empty}
%\NoBgThispage

%----------------------------------------------------------------------------------------
%	INCLUDE IQ GENERAL DESCRIPTION
%----------------------------------------------------------------------------------------
\begin{minipage}{0.35\linewidth} % [b] or [t] for alignment
\includegraphics[width=3.7cm]{images/iq.png}\\
\end{minipage}
%
\begin{minipage}{0.65\linewidth}
\LARGE\color{DarkPurple}\textbf{IQ Digital} 
\large{is a methodology developed by Math Marketing which measures the actual experience offered to users. As a diagnosis tool, it helps both customers and teams to identify potential areas for future development and improvement. Its goal is to optimize engagement and conversion through the combination of 6 distinct dimensions. Market reports as this one, apply this methodology only to publicly available information and provide only high-level diagnosis.}
\end{minipage}

\vspace{1cm}
%----------------------------------------------------------------------------------------
%	INCLUDE EVERY PILLAR DESCRIPTION
%----------------------------------------------------------------------------------------

\setlength{\columnsep}{1cm}
\setlength{\columnseprule}{1pt}
\def\columnseprulecolor{\color{Grey}}
\begin{multicols}{2}
\subsubsection{INTERACTION}
It measure the capacity available for new customer impressions, attraction and interaction.\\

\textbf{\large{Ads}}\\
Since it is not possible to evaluate the competition's digital advertising assertiveness, it looks into current available tools:
\setlist{nolistsep}
\begin{enumerate}[noitemsep]
  \item Uses Paid Ads 
  \item Has a social media strategy
  \item Uses an aggregator
  \item Uses an optimizer
  \item DMP with internal and external data
\end{enumerate}

\textbf{\large{Content}}\\
Measures domain ranking and SEO compared to direct competition from 0 to 5 which means an optimized domain rank and SEO

\textbf{\large{Automation}}\\
Searches online channels for processes which automate the relationship with users and deliver a contextual and taylored experience.
\setlist{nolistsep} 
\begin{enumerate}[noitemsep]
  \item Automation tools are available, but their use is limited 
  \item Tracking tools with journey optimization
  \item Available tools allow conversion and uses progressive profiling
  \item Allows automatic shots adaptable to the user's persona
  \item Allows contextual offer
\end{enumerate}

\subsubsection{PERFORMANCE}
Measures how easy it is to find content or products on a website and the overall speed and structure of the online channel. Low values may imply structural issues which can require an overall site restructure.\\

\textbf{\large{Experience}}\\
Evaluates the search capabilities and access to information on the site: 
\setlist{nolistsep}
\begin{enumerate}[noitemsep]
  \item There is a search option 
  \item Autocomplete, suggested terms, and related content
  \item Search filters are available, and there are different ways of accessing information
  \item Uses semantic complements
  \item Returns results based on context
\end{enumerate}

\textbf{\large{Technical}}\\
Results reflect page performance including page load speed on both desktop and mobile. Higher values imply higher performance. Lower values can seriously compromisse experience specialy when a heavy tag strategy is implemented.

\subsubsection{ANALYTICS}
Checks for performance measurement tools: 
\setlist{nolistsep}
\begin{enumerate}[noitemsep]
  \item Uses an analytics tool for basic metrics 
  \item Uses a tagmanager tool
  \item Uses goals, dimensions and events
  \item Uses an A / B or MVT tool as well as a heatmap
  \item Information is centrally accessible using predictive models
\end{enumerate}
\end{multicols}

\newpage